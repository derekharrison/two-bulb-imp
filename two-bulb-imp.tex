\documentclass[twocolumn]{article}
\usepackage{graphicx}
\usepackage{amsmath}
\usepackage{amssymb}
\usepackage{float}
\makeatletter
\newcommand*\bigcdot{\mathpalette\bigcdot@{.4}}
\newcommand*\bigcdot@[2]{\mathbin{\vcenter{\hbox{\scalebox{#2}{$\m@th#1\bullet$}}}}}
\makeatother
\usepackage{siunitx} % Required for alignment

\sisetup{
  round-mode          = places, % Rounds numbers
  round-precision     = 2, % to 2 places
}

\begin{document}
\author{Derek W. Harrison}
\title{Numerical solution of the Maxwell-Stefan equations using the Finite Volume Method}

\twocolumn[
\begin{@twocolumnfalse}
\maketitle
\begin{abstract}
\noindent
The Maxwell-Stefan equations for multi-component diffusion are solved using the finite volume method. Time discretization is fully implicit. To validate the approach the Maxwell-Stefan equations are used to model the three-component twin-bulb experiment and the results are compared with experimental observations.
\end{abstract}
\end{@twocolumnfalse}
]

\paragraph*{Twin-bulb model}
The twin-bulb experiment [\ref{eqn:my_ref_2}] consists of two small compartments, connected by a tube through which the components can diffuse. The compartments are filled with $H_2$, $N_2$ and $CO_2$. Diffusion through the tube can be modeled using the Maxwell-Stefan equations [\ref{eqn:my_ref_1}]:
\begin{equation}
\label{eqn:eqn_2}
-\left( \frac{\partial \ln{\gamma_i}}{\partial \ln{x_i}} + 1 \right) \nabla x_i = \sum_{j \neq i} \frac{x_j \textbf{J}_i - x_i \textbf{J}_j}{c_t D_{ij}}
\end{equation}
For ideal systems the activity coefficient $\gamma_i$ of component $i$ is equal to unity. The left side of (\ref{eqn:eqn_2}) then simplifies, resulting in:
\begin{equation}
\label{eqn:eqn_3}
- \nabla x_i = \sum_{j \neq i} \frac{x_j \textbf{J}_i - x_i \textbf{J}_j}{c_t D_{ij}}
\end{equation}
From a mass balance follows that the change in local composition at any given time is:
\begin{equation}
\label{eqn:eqn_4}
c_t \frac{\partial x_i}{\partial t} = - \nabla \bigcdot \textbf{J}_i
\end{equation}
Diffusion occurs at constant pressure. To preserve the total concentration the fluxes of the different components sum up to zero:
\begin{equation}
\label{eqn:eqn_5}
\sum_{i} \textbf{J}_i = 0
\end{equation}
\paragraph*{Method}
To compute the composition of the system the model equations (\ref{eqn:eqn_2}) - (\ref{eqn:eqn_5}) are solved using the finite volume method. Time discretization is fully implicit, achieved by eliminating the flux-components from the model equations. Central differencing is used for the diffusion terms.

An explicit method is also explored as a reference to obtain a measure of the stability of the implicit approach.

The mole fractions of $H_2$, $N_2$ and $CO_2$ in the first compartment are initially 0.0, 0.501 and 0.499, respectively. In the second compartment the mole fractions of $H_2$, $N_2$ and $CO_2$ are initially 0.501, 0.499 and 0.0, respectively. The diffusivities are $D_{12} = 8.33e-5$  $(m^2/s)$, $D_{13} = 6.8e-5$ $(m^2/s)$ and $D_{23} = 1.68e-5$ $(m^2/s)$. The volumes of the compartments are $5e-4$ $(m^3)$ and the tube connecting the compartments has a length of $1e-2$ $(m)$ and a diameter of $2e-3$ $(m)$. 

\paragraph*{Results} 
The results of solving the model equations are shown in figure \ref{fig:fig_1}. The results agree well with experimental observations made in [\ref{eqn:my_ref_2}].

\begin{figure}
\includegraphics[width=\linewidth]{graph_of_results.png}
\caption{The mole fraction as a function of time (h).}
\label{fig:fig_1}
\end{figure}

\paragraph*{Discussion}
Explicit time discretization was only stable for small timesteps, whereas the implicitly discretized approach was stable even for large timesteps. Analysis of the coefficients of the linear system obtained from implicit discretization of the model equations shows that the coefficients satisfy the Scarborough criterion, and so the matrix associated with the linear system is diagonally dominant.

\paragraph*{Conclusion}
The Maxwell-Stefan equations were solved using the finite volume method. Results obtained numerically were shown to agree well with experimental observations, thereby validating the model. Regarding the discretization strategy, it was found that implicit time discretization leads to more stable solutions compared with explicit schemes. 

\paragraph*{Appendix}
Here the one-dimensional case of the Maxwell-Stefan equations is elaborated on. The one-dimensional case of (\ref{eqn:eqn_3}) for component $1$ is:
\begin{equation}
\label{eqn:eqn_6}
-c_t \frac{\partial x_1}{\partial z} = \frac{x_2 J_1 - x_1 J_2}{D_{12}} + \frac{x_3 J_1 - x_1 J_3}{D_{13}}
\end{equation}
The equation for component $2$ is:
\begin{equation}
\label{eqn:eqn_7}
-c_t \frac{\partial x_2}{\partial z} = \frac{x_1 J_2 - x_2 J_1}{D_{12}} + \frac{x_3 J_2 - x_2 J_3}{D_{23}}
\end{equation}
The change in local composition of component $1$ is:
\begin{equation}
\label{eqn:eqn_8}
ct \frac{\partial x_1}{\partial t} = - \frac{\partial J_1}{\partial z}
\end{equation}
The change in local composition of component $2$ is:
\begin{equation}
\label{eqn:eqn_9}
ct \frac{\partial x_2}{\partial t} = - \frac{\partial J_2}{\partial z}
\end{equation}
To facilitate the elimination of the fluxes from the equations above equation (\ref{eqn:eqn_6}) and (\ref{eqn:eqn_7}) are rewritten:
\begin{equation}
\label{eqn:eqn_10}
ct \frac{\partial x_1}{\partial z} = a_1 J_1 + a_2 J_2
\end{equation}
\begin{equation}
\label{eqn:eqn_11}
ct \frac{\partial x_2}{\partial z} = b_1 J_1 + b_2 J_2
\end{equation}
With $a_1$, $a_2$, $b_1$ and $b_2$ given by:
\begin{equation}
\label{eqn:eqn_12}
a_1 = \left( \frac{1}{D_{12}} - \frac{1}{D_{13}} \right) x_2 + \frac{1}{D_{13}}
\end{equation}
\begin{equation}
\label{eqn:eqn_13}
a_2 = \left( \frac{1}{D_{13}} - \frac{1}{D_{12}} \right) x_1
\end{equation}
\begin{equation}
\label{eqn:eqn_14}
b_1 = \left( \frac{1}{D_{23}} - \frac{1}{D_{12}} \right) x_2
\end{equation}
\begin{equation}
\label{eqn:eqn_15}
b_2 = \left( \frac{1}{D_{12}} - \frac{1}{D_{23}} \right) x_1 + \frac{1}{D_{23}}
\end{equation}
The fluxes can now be written in terms of the composition gradients:
\begin{equation}
\label{eqn:eqn_16}
J_1 = \beta_1 \frac{\partial x_1}{\partial z} + \beta_2 \frac{\partial x_2}{\partial z}
\end{equation}
\begin{equation}
\label{eqn:eqn_17}
J_2 = \alpha_1 \frac{\partial x_1}{\partial z} + \alpha_2 \frac{\partial x_2}{\partial z}
\end{equation}
With $\beta_1$, $\beta_2$, $\alpha_1$ and $\alpha_2$ given by:
\begin{equation}
\label{eqn:eqn_18}
\beta_1 = -\frac{c_t}{a_1} - \frac{a_2 \alpha_1}{a_1}
\end{equation}
\begin{equation}
\label{eqn:eqn_19}
\beta_2 = - \frac{a_2 \alpha_2}{a_1}
\end{equation}
\begin{equation}
\label{eqn:eqn_20}
\alpha_1 = - \frac{c_t}{a_2 - \frac{a_1 b_2}{b_1}}
\end{equation}
\begin{equation}
\label{eqn:eqn_21}
\alpha_2 = \frac{a_1 c_t}{a_2 b_1 - a_1 b_2}
\end{equation}

\section*{Nomenclature}
\begin{table}[H]
    \begin{tabular}{c l}       
      $c_t$ & Total concentration ($mol \cdot m^{-3}$) \\
      $D_{ij}$ & Diffusivity ($m^2 \cdot s^{-1}$) \\
      $\textbf{J}_i$ & Flux vector ($mol \cdot m^{-2} \cdot s^{-1}$) \\
      $J_i$ & Flux component ($mol \cdot m^{-2} \cdot s^{-1}$) \\   
      $R$ & Universal gas constant (8.314 $J \cdot mol^{-1} \cdot K^{-1}$) \\  
      $T$ & Temperature ($K$) \\       
      $x_i$ & Mole fraction (-)\\
      $\gamma_i$ & Activity coefficient (-) \\
      $z$ & Axial coordinate ($m$) \\
    \end{tabular}
\end{table}

\subsection*{Subscripts}
\begin{table}[H]
    \begin{tabular}{c l} 
      $i$ & Component index (-)\\
      $j$ & Component index (-) \\
    \end{tabular}
\end{table}

\begin{thebibliography}{5}
\bibitem{Duncan}
\label{eqn:my_ref_2}
Duncan, J.B., Toor, H.L. \textit{AIChE J.}, 1962, \textbf{8}, 38–41.
\bibitem{Taylor} 
\label{eqn:my_ref_1}
Taylor, R., Krishna, R. (1993). \textit{Multicomponent Mass Transfer}. New York: Wiley.
\end{thebibliography}

\end{document}